\documentclass{beamer}

\usetheme{Madrid}
\usecolortheme{beaver}
\setbeamertemplate{caption}[numbered]
\makeatother
\setbeamertemplate{footline}
{\leavevmode%
  \hbox{%
  \begin{beamercolorbox}[wd=.4\paperwidth,ht=2.25ex,dp=1ex,center]{author in head/foot}%
      \usebeamerfont{author in head/foot}\insertshortauthor{}~(\insertshortinstitute{})
  \end{beamercolorbox}%
  \begin{beamercolorbox}[wd=.5\paperwidth,ht=2.25ex,dp=1ex,center]{title in head/foot}%
      \usebeamerfont{title in head/foot}\insertshorttitle{}
  \end{beamercolorbox}%
  \begin{beamercolorbox}[wd=.1\paperwidth,ht=2.25ex,dp=1ex,center]{date in head/foot}%
    \insertframenumber{} / \inserttotalframenumber\hspace*{1ex}
  \end{beamercolorbox}}%
}
\makeatletter
\setbeamertemplate{navigation symbols}{}

\usepackage[utf8]{inputenc}
\usepackage[english, russian]{babel}

\usepackage{animate}
\usepackage{graphicx}
\bibliographystyle{unsrt}

\usepackage{hyperref}
\hypersetup{colorlinks=true,
    linkcolor=blue,
    filecolor=magenta,
    urlcolor=cyan,
}

\usepackage{amssymb}
\usepackage{amsmath}
\usepackage{amsthm}
\usepackage{lscape}
\usepackage{color}
\graphicspath{{./images/}}

\usepackage{animate}


%Information to be included in the title page:
\title[Обратная задача, органическая резестивная память]{Обратная задача определения параметров ячейки органической резистивной памяти}
\author[А.А. Борзунов]{А.А.~Борзунов~\inst{1} \and Д.В.~Лукьяненко~\inst{1}  \and А.Г.~Ягола~\inst{1} \and М.С.~Котова~\inst{2}}
\institute[МГУ им М.В. Ломоносова]{\inst{1}каф. математики, физический факультет, Московский Государственный Университет им. М.В. Ломоносова \and \inst{2}каф. общей физики и физики конденсированного состояния, физический факультет, Московский Государственный Университет им. М.В. Ломоносова}
\titlegraphic{\includegraphics[width=2cm]{ff.png}}

\date{24 Сентября 2019 г.}



\begin{document}

\begin{frame}
    \titlepage{}
\end{frame}

\begin{frame}
    \frametitle{Ячейка резистивной памяти}
    \begin{figure}
        \includegraphics[width=0.65\linewidth]{UT_static.eps}
        \caption{Слева~--- распределения потенциала, справа~--- температуры внутри ячейки памяти.}
    \end{figure}
\end{frame}

\begin{frame}
    \frametitle{Коллапс проводящего канала}
    \begin{figure}
        \animategraphics[loop,controls,width=0.6\linewidth]{5}{UT-}{0}{19}
        \caption{Разрушение проводящего канала.}
    \end{figure}
\end{frame}

\begin{frame}[allowframebreaks]
    \frametitle{Модель}
    \begin{enumerate}
        \item Вычисляем $\rho(y,r)$:
        \begin{equation*}
            \rho(y,r) =
            \left\{
            \begin{aligned}
                &\rho_0(y,r), &  &r < R(y,t),\\
                &\rho_m, & &r \geqslant R(y,t),
            \end{aligned}
            \right.
        \end{equation*}
        где $ \rho_0(y,r)=\rho_{00}\Big(1 + \alpha \big(T(y,r) - T_0\big) \Big)~\cite{russo2008}$.

    \item Решаем эллиптическую задачу для потенциалов $V(y,r,t)$ ($t$~--- параметр):
        \begin{equation*}
            \left\{
            \begin{aligned}
                &\nabla \left(\frac{1}{\rho}\nabla V\right) = 0, \quad y \in (0,l), \quad r \in [0,l_r],\\
                &V(0,r,t) = 0, \quad V(l_y,r,t) = V_a,\\
                &\dfrac{\partial}{\partial r}V(y,0,t) = \dfrac{\partial}{\partial r} V(y,l_r,t)=0,
            \end{aligned}
            \right.
        \end{equation*}

    \item Вычисляем компоненты вектора плотности тока $\vec{J}$ $\vec{J}(y,r,t) =1/\rho \nabla V$, и $J^2 \equiv |\vec{J}|^2$ в каждой точке рассматриваемой области.

    \item Решаем эллиптическую задачу для температуры $T(y,r,t)$ ($t$~--- параметр):
        \begin{equation*}
            \left\{
            \begin{aligned}
                &-\nabla (k\nabla T)= \rho J^2, \quad y \in (0,l), \quad r \in [0,l_r],\\
                &T(0,r,t) = 300, \quad T(l_y,r,t) = 300,\\
                &\dfrac{\partial}{\partial r}T(y,0,t) = \dfrac{\partial}{\partial r} T(y,l_r,t)=0.
            \end{aligned}
            \right.
        \end{equation*}

    \item Вычисляем координаты проводящего канала $R(y,T,t)$ в новый момент времени:
        \begin{equation*}
            R(y,t + \tau) = R(y,t) - v_{D0} e^{-E_a/{k}_{BT}} \tau~\cite{russo2008}.
        \end{equation*}

    \item Задаем новый момент времени $t: = t + \tau$. Если $t < \theta$, то переходи к пункту~2.

    \end{enumerate}
\end{frame}

\begin{frame}
    \frametitle{Минимизация функционала}
    Минимизируемый функционал представляет собой квадрат невязки экспериментальной и полученной в ходе численного эксперимента вольт-амперной характеристики $ J(\rho,\alpha,k) = || I_m - I_e||^2 $
    \begin{figure}
        \includegraphics[width=0.55\linewidth]{IV.jpg}
        \caption{Синяя кривая~--- экспериментальные данные, черная~--- результат моделирования с заданными параметрами.}
    \end{figure}
\end{frame}

\begin{frame}
    \frametitle{Эволюционная факторизация эллиптического уравнения}

    Следующее приближение $\hat{u}$ функции $u$ через момент времени $\tau$ будет определяться из  факторизованного уравнения
    \begin{equation*}
        \Big(E - a\tau \Lambda_y\Big)\Big(E - a \tau \Lambda_r\Big)\dfrac{\hat{u} - u}{\tau} = (\Lambda_r + \Lambda_y) u + f\Big(y,r,t+\dfrac{\tau}{2}\Big),
    \end{equation*}
    где
    \begin{align*}
    &{\Big(\Lambda_r u\Big)}_{m,n} = \dfrac{1}{dr}\bigg[\frac{1}{\rho(y_{m},r_{n + 1/2})}\dfrac{u_{m,n + 1} - u_{m,n}}{dr} - \frac{1}{\rho(y_{m},r_{n - 1/2})}\dfrac{u_{m,n} - u_{m,n - 1}}{dr}\bigg], \\
    &{\Big(\Lambda_y u\Big)}_{m,n} = \dfrac{1}{dy}\bigg[\frac{1}{\rho(y_{m + 1/2},r_{n})}\dfrac{u_{m + 1,n} - u_{m,n}}{dy} - \frac{1}{\rho(y_{m - 1/2},r_{n})}\dfrac{u_{m,n} - u_{m - 1,n}}{dy}\bigg];
\end{align*}
или, в другом виде:
\begin{align*}
    &\Big(E - a\tau \Lambda_y\Big)v = (\Lambda_r + \Lambda_y) u + f\Big(y,r,t+\dfrac{\tau}{2}\Big), \\
    &\Big(E - a \tau \Lambda_r\Big)\dfrac{\hat{u} - u}{\tau} = v.
\end{align*}

\end{frame}

\begin{frame}
    \begin{figure}
        \includegraphics{FACT.eps}
        \caption{Схематическое изображение решения эллиптической задачи с помощью эволюционной факторизации.}
    \end{figure}
\end{frame}


\begin{frame}[allowframebreaks]
    \bibliography{bibliography}
\end{frame}

\begin{frame}
    \frametitle{Спасибо за внимание!}
    Текст презентации доступен по ссылке:
    \url{https://github.com/aborzunov/report24Sep}
    \begin{figure*}
        \includegraphics[width=0.5\linewidth]{repo_link.eps}
    \end{figure*}
    \usebeamerfont{institute} Contact: \url{Andrey.Borzunov@gmail.com}

\end{frame}
\end{document}
